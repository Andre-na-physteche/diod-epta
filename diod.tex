\documentclass[a4paper]{article}
\usepackage[utf8]{inputenc}
\usepackage[russian]{babel}
\usepackage[T2]{fontenc}
\usepackage[warn]{mathtext}
\usepackage{graphicx}
\usepackage{amsmath}
\usepackage{floatflt}
\usepackage[left=20mm, top=20mm, right=20mm, bottom=20mm, footskip=10mm]{geometry}


\graphicspath{ {images/} }
\usepackage{multicol}
\setlength{\columnsep}{2cm}


\begin{document}

\begin{titlepage}
	\centering
	\vspace{5cm}
	{\scshape\LARGE Московский физико-технический институт \par}
	\vspace{5cm}

	{\huge\bfseries Термоэлектронный диод \par}
	\vspace{1cm}
	{\scshape\Large Лабораторная работа по курсу <<Вакуумная электроника>>\par}
	\vspace{1cm}
	\vfill
\begin{flushright}
	{\large выполнила студентка 004 группы ФЭФМ}\par
	\vspace{0.3cm}
	{\LARGE Богатова Екатерина Александровна} \par

	
\end{flushright}
	

	\vfill

% Bottom of the page
	Долгопрудный, 2021 г.
\end{titlepage}

\section{Цель работы}
Изучить на практике явления термоэлектронной эмиссии и процессов токопрохождения в вакууме; изготовить вакуумный диод; исследовать некоторые характеристики диода; проверить справедливость законов Ричардсона-Дешмана и Чайлда-Ленгмюра.

\section{Лабораторная установка}
Схема лабораторной установки для исследования характеристик термоэлектронного диода приведена на рис. 1.

\begin{figure}[h]
    \centering
    \includegraphics[width=13cm]{setup.png}
    \caption{Принципиальная схема лабораторной установки}
    \label{fig:vac}
\end{figure}
\begin{enumerate}
    \item Форвакуумный насос
\item Турбомолекулярный насос
\item Вакуумная камера
\item Клапан с электрическим управлением
\item Измерительная насадка
\item Фильтр входящего воздуха
\item Диод
\item Источник питания HY 3010E
\item Вольтметр GPR-30H100
\end{enumerate}

\section{Изготовление диода}
\subsection{Изготовление анода}
\begin{itemize}
    \item Для изготовления анода используем никелевую пластину $30 \cdot 40~\text{мм}$. Для большей жесткости анода на заготовке с помощью шила проводятся канавки (ребра жесткости).
    \item Заготовка наматывается на оправку с захлестом $3~\text{мм}$, а края пластины завальцовываются на шве для более плотного прилегания.
    \item Сварка производится с шагом $3~\text{мм}$ межу точками.
\begin{figure}[h]
\begin{center}
\includegraphics[width=13cm]{заготовка анода.jpg}
\end{center}
\end{figure}
\end{itemize}
\subsection{Монтаж анода на ножку}
\begin{itemize}
    \item Изготавливаем траверсы из двух отрезков никелевой проволоки по $50~\text{мм}$ каждый. Проволока прокатывается между двух пластин для выравнивания и профилируется.
    \item Траверсы привариваются к аноду так, чтобы ось анода находилась строго между ними.
\begin{figure}[h]
\begin{center}
\includegraphics[width=13cm]{траверса анода.jpg}
\end{center}
\end{figure}
    \item Далее анод монтируется на ножу так, как показано на рисунке ниже. Предварительно надо очистить выводы ножки от окисла при помощи надфиля.
\end{itemize}
\begin{figure}[h]
\begin{center}
\includegraphics[width=13cm]{посадка на ножку.jpg}
\end{center}
\end{figure}
\newpage
\subsection{Подготовка крепления катода}
\begin{itemize}
    \item Для изготовления катода необходимы никелевая проволока длиной $80~\text{мм}$ и две полоски из никела размерами $2 \cdot 20~\text{мм}$ и  $2 \cdot 10~\text{мм}$. Для изготовления самого катода понадобится отрезок вольфрамовой проволоки длиной $50~\text{мм}$.
    \item Сначала изготавливается траверса и два крепления катода, затем на конец натягивающей траверсы приваривается первое крепление и сама траверса приваривается на соответствующий вывод ножки (см. рисунок)
\begin{figure}[h]
\begin{center}
\includegraphics[width=13cm]{натягивающая траверса катода.jpg}
\end{center}
\end{figure}
\end{itemize}
\subsection{Монтаж катода}
\begin{itemize}
    \item Сначала отрезок вольфрамовой проволоки распрямляется, затем катод закрепляется во втором креплении.
\begin{figure}[h]
\begin{center}
\includegraphics[width=13cm]{заварка катода.jpg}
\end{center}
\end{figure}
    \item Далее крепление вместе с катодом приваривается на соответствующий вывод ножки.
    \item Происходит натягивание и фиксация катода: для этого свободный конец катода вкладывается в согнутое крепление на натягивающей траверсе. Затем катод натягивается и его приваривают к креплению.
\end{itemize}

\newpage
\section{Выполнение работы}
\begin{enumerate}
    \item Проведём прогревание катода, снимем зависимость тока накала от напряжения накала. График зависимости приведён на рисунке 2.
   
\item Рассчитаем сопротивление диода по формуле 
\begin{center}
    $R = \frac{U}{I}$,
\end{center}

подаваемую мощность - по формуле

\begin{center}
    $P = U I$
\end{center}

Построим график зависимости сопротивления катода от приложенной мощности (рисунок 3)



\item Построим графики зависимости температуры катода от тока накала.  \par
Для графика, построенного на основании изменения сопротивления катода, температуру будем вычислять по формуле, полученной в результате преобразований:
\begin{center}
   $ \rho_T = \rho_0 (1 + \alpha T) $, \\
   $T = \frac{\rho_T - \rho_0}/{\alpha \rho_0}$
\end{center}
где $\alpha = 9,29 * 10^{-3}$ - коэффициент температурной зависимости электрического сопротивления, $\rho_0$ - удельное сопротивление материала катода при $0^{\circ}$. \par

Для графика, построенного на основании расчётов с использованием уравнения энергетического баланса, используем формулу 
\begin{center}
    $T_k = (\frac{P}{\varepsilon S \sigma})^{1/4}$,
\end{center}
где $S$ - площадь эмитирующей поверхности, $\varepsilon = 0.032$ - степень черноты материала катода, $\sigma = 5.67*10^{-8}$ Дж/(с*м$^2$*K$^4$). \par
На одном графике представим эти зависимости, их характеры совпадают.



\item Построим графики зависимости анодного тока от анодного напряжения пи различных значениях тока накала в координатах $lg(I_A)$ от $lg(U_A)$



\item По этим данным определим первеанс диода $g$. Зная, что $I_a = gV_A^{3/2}$ и имея зависимости $\lg(I_A) = k \lg(V_A) + b$, получим
\begin{center}
    $g = 10^{\frac{3b}{2k}}$
\end{center}

Определим первеанс по разным данным тока накала диода и сравним значения с расчётным, вычисляющимся по формуле

\begin{center}
    $g = 2.33*10^{-6} \frac{S_c}{R_a ^2} = 3.4485*10^{-3}$,
\end{center}
где $S_c$ - площадь поверхности катода, $R_a$ - радиус анода. \par

Также вычислим отношение заряда электрона к его массе по формуле

\begin{center}
   $e/m = \frac{81}{8}(g\frac{R_a}{L_a})^2$
\end{center}

Наконец, эффективность катода вычислим по формуле
\begin{center}
    $\eta = \frac{I}{P}$
\end{center}

    \begin{table}[h]
    \centering
    \begin{center}
    \caption{Значения первеанса по разным опытам}
    \end{center}
    \vspace{0.1cm}
    \label{tab:my_label}
    \begin{tabular}{ |p{1cm}|p{1cm}|p{1cm}|p{2cm}|p{2cm}|p{2cm}|p{2cm}|}
 \hline
 $I$, А & $U$, B & k & b & g & e/m & $\eta$, \% \\
 \hline
2.4 & 4.7 & 0.137 & -4,4898 & 6.944*$10^{-50}$ & -- & 21.3\\
 \hline
2.5 & 5.3 & 0.2409 & -4,0071 & 1.12*$10^{-25}$ & 3.528*$10^{-51}$ & 18.7\\
 \hline
2.6 & 5.5 & 0.3184 & -3,7983 & 1.27*$10^{-18}$ & 4.536*$10^{-37}$ & 18.0\\
 \hline
2.7 & 6.0 & 0.5798 & -3,7735 & 1.73*$10^{-10}$ & 8.41*$10^{-21}$ & 16.7\\
 \hline
2.8 & 6.3 & 0.7506 & -3,7191 & 3.69*$10^{-8}$ & 3.83*$10^{-16}$ & 15.9 \\
 \hline
2.9 & 6.6 & 1.0087 & -3,8266 & 2.04*$10^{-6}$ & 1.17*$10^{-12}$ & 15.0\\
 \hline
3.0 & 7.0 & 1.2219 & -3,9186 & 1.55*$10^{-5}$ & 6.75*$10^{-11}$ & 14.3\\
 \hline 
\end{tabular}
\end{table}

\item Построим зависимость анодного тока от тока накала при напряжении V = 110 B

\end{enumerate}

\section{Вывод}

В ходе лабораторной работы
\begin{enumerate}
    \item Были получены представления о структуре элементарного диода

    \item Были изучены следующие характеристики диода: вольт-амперная характеристика, первеанс и его эффективность;

    \item Были проверены закономерности ВАХ диода: при больших токах накала - справедливо уравнение Чайлда-Ленгмюра, а при насыщении – уравнение Ричардсона-Дэшмана;

    \item Была рассчитана температура катода, исходя из трёх разных позиций: с точки зрения сопротивления катода, с точки зрения уравнения энергетического баланса, с точки зрения закона Ричардсона-Дэжзшмана;

\end{enumerate}

\section{Список литературы}
\begin{enumerate}
    \item Батурин А.С., Кириченко Л.А., Коновалов Н.Д. и др. Под ред. Шешина Е.П. Эмиссионная электроника в примерах и задачах: учебное пособие / --М.:МФТИ, 2002 - 193 с. 
    \item Батурин А.С., Стариков П.А., Шешин Е.П. Термоэлектронный диод: лабораторная работа по курсу Вакуумная электроника /--М.:МФТИ, 2008 - 43 с.
\end{enumerate}

\newpage

\begin{figure}[h]
\begin{center}
\includegraphics[width=13cm]{fig1.png}
\caption{Зависимость тока накала от напряжения накала}
\end{center}
\end{figure}

\begin{figure}[h]
\begin{center}
\includegraphics[width=13cm]{fig2.png}
\caption{Зависимость сопротивления катода от приложенной мощности}
\end{center}
\end{figure}

\begin{figure}[h]
\begin{center}
\includegraphics[width=13cm]{fig3.png}
\caption{Зависимость температуры катода от тока накала: расчёт по измерению сопротивления и по уравнению Стефана-Больцмана}
\end{center}
\end{figure}

\begin{figure}[h]
\begin{center}
\includegraphics[width=13cm]{2_4.png}
\caption{Зависимость $lg(I_A)$ от $lg(V_A)$ при токе накала 2.4 А, напряжении накала 4.7 B}
\end{center}
\end{figure}

\begin{figure}[h]
\begin{center}
\begin{minipage}[h]{0.45\linewidth}
\includegraphics[width=1\linewidth]{2_5.png}
\caption{Зависимость $lg(I_A)$ от $lg(V_A)$ при токе накала 2.5 А, напряжении накала 5.3 B}
\end{minipage}
\hfill 
\begin{minipage}[h]{0.45\linewidth}
\includegraphics[width=1\linewidth]{2_6.png}
\caption{Зависимость $lg(I_A)$ от $lg(V_A)$ при токе накала 2.6 А, напряжении накала 5.5 B }
\end{minipage}
\end{center}
\end{figure}

\begin{figure}[h]
\begin{center}
\begin{minipage}[h]{0.45\linewidth}
\includegraphics[width=1\linewidth]{2_7.png}
\caption{Зависимость $lg(I_A)$ от $lg(V_A)$ при токе накала 2.7 А, напряжении накала 6.0 B}
\end{minipage}
\hfill 
\begin{minipage}[h]{0.45\linewidth}
\includegraphics[width=1\linewidth]{2_8.png}
\caption{Зависимость $lg(I_A)$ от $lg(V_A)$ при токе накала 2.8 А, напряжении накала 6.3 B }
\end{minipage}
\end{center}
\end{figure}

\begin{figure}[h]
\begin{center}
\begin{minipage}[h]{0.45\linewidth}
\includegraphics[width=1\linewidth]{2_9.png}
\caption{Зависимость $lg(I_A)$ от $lg(V_A)$ при токе накала 2.9 А, напряжении накала 6.6 B}
\end{minipage}
\hfill 
\begin{minipage}[h]{0.45\linewidth}
\includegraphics[width=1\linewidth]{3_0.png}
\caption{Зависимость $lg(I_A)$ от $lg(V_A)$ при токе накала 3.0 А, напряжении накала 7.0 B }
\end{minipage}
\end{center}
\end{figure}

\begin{figure}[h]
\begin{center}
\includegraphics[width=13cm]{fig4.png}
\caption{Зависимость $lg(I_A)$ от тока накала при V = 110 B}
\end{center}
\end{figure}

\end{document}
